\maketitle
\section{Learning the basics - Mathematical mode}
- Math mode: 
$(x+3+4)$ \\
- Displayed math mode: %{\hrule height 1pt}
$$A(x)=x^2+2x+4$$
- Superscript: 
$$ 2x^{34+5x}$$
- Subscript: 
$$x_1$$
$$x_{12}$$
- Sub-subscripts:
$$x_{1_3}$$
$$a_0, a_1, a_2, \ldots, a_{100}$$
- Greek letters
$$\pi$$
$$\Pi$$
$$\alpha$$
$$A=\pi r^2$$
- Trig functions:
$$y=\sin x$$
$$y=\cos x$$
$$y=\csc \theta$$
$$y=\sin^{-1} x$$
$$y=\arcsin x$$
- Log functions:
$$y=\log x$$
$$y=\log_5 x$$
$$y=\ln x$$
-Roots \& square roots:
$$\sqrt{2}$$
$$\sqrt[3]{2}$$
$$\sqrt{x^2+y^2}$$
$$\sqrt{   1+\sqrt{x}     }$$
-Simple Fractions:
$$\frac{2}{3}$$
-More complicated fractions: 
$$\frac{\sqrt{x+1}}{\sqrt{x+2}}$$
$$\frac{\sqrt{x+3}}{\sqrt{x+4}}$$
$$\frac{1}{      (1+\frac{1}{x})       }$$
-Sizing the fractions:

· Example: $\displaystyle \frac{1}{2}$ of the population is...\\[10pt]%space between these sentences [10pt of space]

· Example: $\frac{2}{3}$ of the population is...\\[10pt]

· Example: $\dfrac{2}{3}$ of the population is... (we need a package for this)\\
-Arrays:
\begin{align}
5x^2-9=x+3\\
5x^2-x-12=0
\end{align}

%to line the equal sign you add &before the =
\begin{align*}  
5x^2-9&=x+3\\
5x^2-x-12&=0\\
0&=12+x-5x^2
\end{align*}

\begin{align}
5x^2-9=x+3\\
5x^2-x-12=0
\end{align}
