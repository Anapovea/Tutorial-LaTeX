\maketitle
\section{Tables}
%%%%%%%%%%%%%%%%% FIRST TABLE %%%%%%%%%%%%%%%%% 
Basic table:\\

\begin{tabular}{|c||c|c|c|l|r|}  %this means that the content is going to be align in the Center. C=center, L=left, r=r
\hline
$x$ & 1 & 2 & 3 & 4 & 5 \\ \hline
$g(x)$ & 10 & 11 & 12 & 13 & 14 \\ \hline
\end{tabular}

%%%%%%%%%%%%%%%%% SECOND TABLE %%%%%%%%%%%%%%%%% 
%adding the float package. 
\begin{table}[H]
\centering
\def\arraystretch{1.5} %ADDING SPACE IN THE TOP AND THE BOTTOM OF THE TABLE
\begin{tabular}{|c||c|c|c|c|c|}
\hline
$x$ & 1 & 2 & 3 & 4 & 5 \\ \hline
$f(x)$ & $\frac{1}{2}$ & 11 & 12 & 13 & 14 \\ \hline
\end{tabular}
\caption{These values represent the function $f(x)$.}
\end{table}

%%%%%%%%%%%%%%%%% THIRD TABLE %%%%%%%%%%%%%%%%% 

\begin{table}[H]
\centering
\caption{The relationship between $f$ and $f'$.}
\def\arraystretch{1.5}
\begin{tabular}{|l|p{5cm}|} %defining the size of the column
\hline
$f(x)$ & $f'(x)$ \\ \hline
$x>0$ & The function $f(x)$ is increasing. The function $f(x)$ is increasing. The function $f(x)$ is increasing. The function $f(x)$ is increasing. The function $f(x)$ is increasing. \\ \hline
\end{tabular}
\end{table}